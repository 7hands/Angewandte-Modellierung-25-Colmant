\documentclass{scrartcl}
\usepackage{amsmath,amsfonts,amsthm,bm,graphicx}
\usepackage{tikz,pgfplots}
\usepackage{listings}
\usepackage{stmaryrd}
\usepackage{xcolor}
\usepackage{fdsymbol}
\usepackage{rotating}
\usepackage{listings}

\pgfplotsset{width=15cm,compat=1.18}
\allowdisplaybreaks
\setlength{\parindent}{0pt}

\title{Exercize 1: Matlab Basics}
\subtitle{Angewandte Modellierung 25}
\author{Carl Colmant}
\date{\today}
\begin{document}
\maketitle
\section*{Differential Equation}

\subsection*{a)}
Gegeben ist eine Differentialgleichung:
\begin{equation*}
    x''+\omega^2x=0 , \omega>0
\end{equation*}
mit den Randbedingungen:
\begin{equation*}
    x(0)=0,\quad x'(0)=1
\end{equation*}
Eine Möglichkeit ist die Verwendung von \texttt{dsolve} in Matlab. Womit Differentialgleichungen mit gegebenen Bedingungen gelöst werden können.\\
Somit kann die Differentialgleichung in Matlab wie folgt gelöst werden:
\begin{lstlisting}[language=Matlab, caption=Symbolic Math]
    %Symbolic Math:
    syms x(t) omega positive % Symbole

    eq = diff(x, t, 2) + omega^2 * x == 0; % Equation

    Dx = diff(x, t); % first differential
    % Conditions for the Equation
    cond = [x(0) == 0, Dx(0) == 1]; 

    sol = dsolve(eq, cond);
\end{lstlisting}

Eine andere Möglichkeit ist die Verwendung von \texttt{ode45} in Matlab. Womit Differentialgleichungen mit gegebenen Bedingungen numerisch gelöst werden können.\\
Dazu muss man die symbolische Funktion in eine numerische Funktion umwandeln und einen Werte Bereich für die x Achse definieren. Außerdem muss für $\omega$ ein wert eingesetzt werden \\
\begin{lstlisting}[language=Matlab, caption=Numeric Math]
    % numeric Math:
    % Parameter
    omega_val = 1;  % Setze z.B. omega = 2

    % System 1. Ordnung: x1 = x, x2 = dx/dt
    f = @(t, y) [y(2); -omega_val^2 * y(1)];

    % Anfangswerte: x(0) = 0, dx/dt(0) = 1
    y0 = [0; 1];

    % Zeitintervall
    tspan = [0, 5];

    % Numerische Lösung
    [t_num, y_num] = ode45(f, tspan, y0);
    \end{lstlisting}

\subsection*{b)}
Gegeben ist die Differentialgleichung:
\begin{equation*}
    x'=-\alpha x,  \alpha>0
\end{equation*}
mit den Randbedingungen:
\begin{equation*}
    x(0)=1
\end{equation*}


\end{document}