\documentclass{scrartcl}
\usepackage{amsmath,amsfonts,amsthm,bm,graphicx}
\usepackage{tikz,pgfplots}
\usepackage{listings}
\usepackage{stmaryrd}
\usepackage{xcolor}
\usepackage{rotating}
\usepackage{listings}
\usepackage{hyperref}

\pgfplotsset{width=15cm,compat=1.18}
\allowdisplaybreaks
\setlength{\parindent}{0pt}

\title{Assignment 7}
\subtitle{Angewandte Modellierung 25}
\author{Carl Colmant}
\date{\today}
\begin{document}
\maketitle
\newpage
\section*{Exercise 1. }
Ich habe mich für die Bücher "Also Sprach Zarathustra" und "Die Schatzinsel" entschieden in Deutscher original Sprache (ein großer Fehler). Ich habe direkt zwei gewählt weil ich später an probleme gestoßen bin und ich so die Möglichkeit habe zu vergleichen ob es an der Sprache liegt oder an den Texten.

\subsection*{2. }
Zu erst habe ich die text dateien in R studio geladen. Zun Dann habe ich die Stoppwords raus genommen heir für kann man die sprache der erkannten stopp wörter angeben. Da zu kommt aber noch dass die Texte beide sehr alt sind (Die Schatzinsel ist von 1883 und Also Sprach Zarathustra auch von 1883) das führt dazu dass stopp wörter vor kommen die heute nicht mehr gebräuchlich sind und somit auch nicht in der Stoppwörter liste sind. ("dass", "diess", "’s", "wer", "—", "the", "konnt", "ganz", "mehr", "sagt","gutenberg™","project"). Nun entferne ich befor ich die Stoppwords aus dem Text entferne alle kommata, punkte, gänsefüße usw. und änder den Text in Kleinbuchstaben.

\subsection*{3. }
Nun werden noch die Frequenzen der Wörter gezählt und dann kann auhc schon die Wordcloud erstellt werden. Daraus ergeben sich dann folgende Frequency Histogramme und Wordclouds:\\
\includegraphics*[scale=0.3]{freqHistSchatzinsel.png}
\includegraphics*[scale=0.3]{freqHistZarathustra.png}\\
Wie man sieht ist bei der Schatzinsel immer noch ein hochkomma im Text enthalten, das passiert weil das ein extra sonderzeichen ist welches ich nicht gefunden habe. In zarathustra gibt es das gleiche Problem mit dem Wort "--" welches ich nicht gefunden habe.\\
\includegraphics*[scale=0.6]{SchatzinselWordcloud.png}\\
\includegraphics*[scale=0.6]{zarathustraWordcloud.png}\\
Wie man sieht beschäftigt sich die Schatzinsel als Roman vor allem mit seinen Charaktären und every day life so zu sagen. Zarathustra hingegen ist ein philosophisches Buch mit philosophischen Themen z.B. Leben und Tod und dem Geist und der Seele.

\subsection*{4.}
Nun kommen wir aber an Probleme die sich ab hier durch die Aufgaben ziehen, Sentiment analysis ist sehr komplex und die modelle die es gibt sind oft für die englische Sprache trainiert. daher bekommt man hier:\\
\includegraphics*[scale=0.5]{SentimentSchatzinsel.png}\\
\includegraphics*[scale=0.5]{sentimentZarathustra.png}\\
Man sieht das sich die Stimmung der Texte kaum unterscheidet und kaum ausschlägt das ist besonders deswegen komisch weil Zarathustra ein mitreißendes Buch was mit vielen Emotionen geschrieben ist.


\section*{Exercise 2. }
\subsection*{1.}
Ich habe die Texte von davor wiederverwendet und also ein 'Genre Mashup' gemacht. Dazu habe ich in python folgende Bibliotheken benutzt:\\
\includegraphics*[scale=0.25]{libs.png}\\
\subsection*{2.}
Nun müssen die texte für die nächsten Aufgabe vorbereitet werden. Dazu werden stoppwords entfernt.\\
\includegraphics*[scale=0.25]{cleanup.png}\\
\subsection*{3.}
Nun soll die Vocabular der beiden Texte verglichen werden das habe ich mit folgendem Code gemacht.\\
\includegraphics*[scale=0.25]{freq.png}\\
\includegraphics*[scale=0.25]{freq2.png}\\
Hieraus entsteht dann die Ausgabe:\\
--- Shared Vocabulary (3104 Wörter) ---
a, ab, abend, abenteuer, aberglauben, abermals, abhang, abide, abkunft, abnehmen, about, abscheuliches, abschied, abwärts, abzeichnete, accept, accepted, accepting, access, accessed, accessible, accordance, ach, achseln, acht, achten, achtete, active, actual, addition, additional, additions, address, addresses, adern, adler, affe, affen, against, aged, agent, agree, agreed, agreement, ah, ahnt, all, alledem, allein, allow, almost, alone, already, alt, alte, altem, alten, alter, alteration, alternate, altes, amen, anbeginn, anblick, and, anfang, angenehm, angenehme, angenehmen, angenehmer, angst, anhöhe, anker, annehmen, antwort, antworten, antwortete, antworteten, any, anyone, anything, anywhere, anzuhören, apfel, appear, appearing, appears, applicable, apply, approach, arbeit, arbeiten, archive, are, arg, arise, arm, arme, armen, armer

--- Unique to Schatzinsel (6063 Wörter) ---
aal, aalglatt, abbrechend, abdrücke, abe, abendbrot, abenden, abendwind, abenteuerlichen, abenteuern, abergläubischen, abfahrenden, abfallen, abfallenden, abfeuern, abflauen, abfuhr, abgebrochen, abgebrochenen, abgebröckelten, abgefangen, abgefeuert, abgehalten, abgehauenen, abgehärmten, abgekühlt, abgelegt, abgelenkt, abgemacht, abgeplattet, abgesandte, abgeschnitten, abgeschrägte, abgeschwindelt, abgesetzt, abgesperrten, abgesplitterter, abgestattet, abgetragenen, abgetrieben, abgezapft, abgezogen, abhalten, abhanges, abhing, abhänge, abklang, ablaufen, ablehnt, ablehnte, ablud, abmurksen, abnahm, abnahmen, abprallen, abraham, abrechnen, abrechnungsbuch, abrede, abreisen, abrutschte, absatz, abscheuliche, abscheulichen, abscheulicher, abschießen, abschließend, abschloß, abschluß, abschneiden, absetzen, absicht, absichtlich, absitzen, absolut, abspringen, abstehen, absteigen, abstimmen, abstoßendere, abständen, abtrift, abtrocknete, abwarten, abwechselnd, abwechslung, abwechslungsreichsten, abweisungen, abwenden, abwesenheit, abzufangen, abzufeuern, abzuhalten, abzukühlen, abzuschießen, abzuschneiden, abzuwenden, achse, achsel, achselhöhle

--- Unique to Zarathustra (6677 Wörter) ---
aas, aasvögel, abbild, abende, abendliche, abendmahl, abendroth, abendröthen, abends, abendwärts, aberwitzige, abführen, abgeflossen, abgegangen, abgegeben, abgehellter, abgekehrt, abgeknabbert, abgelaufnen, abgemerkt, abgenagt, abgeneidet, abgeraubt, abgerichtet, abgesagt, abgeschirrt, abgeschirrtem, abgethan, abgetödtet, abgewandten, abgezeichnete, abglanz, abgrund, abgrunde, abgrundwärts, abgründe, abgründen, abgründlich, abgründlichen, abgründlicher, abgründliches, abgründlichsten, abgünstig, abhelfen, abhold, abkehr, abkehrte, ablaufe, ableiten, ablerne, ablernen, ablernst, ablernte, abraum, absagen, absagte, abschaffen, abschaum, abschiede, abschieds, abschätzen, abseits, abspeisen, absterbende, absterbenden, abthun, abthätet, abtrennst, abtrünnigen, abwendet, abwägen, abzubitten, abzubrechen, abzählte, abzöge, achtbarkeiten, achtet, achtsam, adel, adelt, adlerhaft, adlern, adlers, afrikanisch, after, ahnungsvollen, allda, alleingehens, alleinsein, alleinseins, allerhöchsten, allerlei, allerliebsten, allermeisten, allerzierlichsten, allesammt, allezeit, allgemach, allgenügsame, allgenügsamen\\
\includegraphics*[scale=0.6]{schatztop20.png}\\
\includegraphics*[scale=0.6]{zaraop20.png}\\
\includegraphics*[scale=0.5]{SchatzinselWordcloudpy.png}\\
\includegraphics*[scale=0.5]{zarathustraWordcloudpy.png}\\
\includegraphics*[scale=0.5]{Ffo.png}\\
\subsection*{4.}
In diesem Teil sollten die Texte genauer auf stimmung, brnutzte Wortarten und Objekte wie z.B. Personen. Die Sentiment analysis hat leider nicht funktioniert weil einerseits der text auf Deutsch ist und sich conda geweigert hat bibliotheken zu finden die das ermöglicht hätten wie z.B. textblob\_de. Dennoch habe ich keine mühen gescheut die POS und NER möglichst gut durchzuführen für die NER habe ich sogar ein transformer modell von huggingface benutzt. Hier zuerst mein Code:\\
\includegraphics*[scale=0.25]{POS.png}\\
\includegraphics*[scale=0.25]{NER.png}\\
Das Problem selbst mit nem Transformer Modell ist das ergebnis eher unterdurchschnittlich:\\
POS Ausgabe: 

Analyse für: Also sprach Zarathustra

Top 5 POS-Tags: [('NOUN', 6860), ('PUNCT', 6005), ('PRON', 4537), ('DET', 4479), ('VERB', 3783)]

Analyse für: Schatzinsel

Top 5 POS-Tags: [('PUNCT', 6998), ('NOUN', 5710), ('PRON', 4459), ('VERB', 4207), ('DET', 4178)]

Die NER kann nur wenig identifiezieren:\\
--- HuggingFace NER (Schatzinsel) ---

PER (7): ['Ballantyne', 'Cooper', 'Kingston', 'Livesay', 'R. L. St.', 'S. L. O.', 'Trelawney']

MISC (1): ['amerikanische']

LOC (2): ['Admiral Benbow', 'Zum Admiral Benbow']`

--- HuggingFace NER (Zarathustra) ---

PER (2): ['Friedrich Wilhelm Nietzsche', 'Zarathustra']

MISC (3): ['Amen-Lied', 'Ja-', 'Lied']

Man kann aber erkennen, dass Die Schatzinsel als Roman mehr wert auf das klare vorstellen der Charaktären legt während Also Sprach Zarathustra nicht so direkt die wichtigkeit der Charaktären festlegt. Beispiels weise wird ganz am anfang eine Figur vorgestellt Der Heilige der wharscheinlich Religiöse führer darstellen soll, dieser ist aber ausser für die vorbereitung der Massage des Buches kommplett unwichtig. Im gegensatz ist der Seiltänzer der etwas später getötet wird ein sehr wichtiger Teil der Argumentation für die übermenschen. Diese beiden Charaktäre kann man aber nur sehr schwer überhaupt automatisch finden da sie weder Namen haben noch besonders häufig vorkommen. 

\section*{Exercise 3.}
Ich hatte leider weder Zeit noch besonders viel erfollg mit Altair AI studio gerade mit Deutschen Texten hab ich das nicht hingekriegt.

\section*{\href{https://github.com/7hands/Angewandte-Modellierung-25-Colmant}{Github}}
Wie immer sind alle meine benutzten Dateien auf meinem Github zu finden. 




\end{document}